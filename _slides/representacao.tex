\lecture{Representação de caracteres no computador}{code}

\lecturetitle{\insertlecture}{\course}

\frame{\maketitle}

\section{Texto}

\subsection{Padrão ASCII}

\begin{frame}{Padrão ASCII}
  O padrão \alert{ASCII} ({\em American Standard Code for Information
    Interchange}) foi desenvolvido na década de 1960 para padronizar a
  comunicação de mensagens.

  \begin{itemize}
  \item 8 bits: 7 para codificação e 1 para detecção de erro
  \item 128 códigos: 95 gráfico e 33 de controle
  \end{itemize}

\end{frame}

\begin{frame}{Código gráfico}{ASCII}

  Parte da tabela ASCII para representação de algumas letras.
  \bigskip
  \footnotesize
\begin{columns}
\begin{column}{.5\textwidth}
  \begin{tabular}{|c|c|c|c|c|}\hline
    \bf bin &	\bf oct &\bf	dec &\bf	hex &\bf	letra\\\hline
  0100 0001 &	101 &	65 &	41 &	A\\\hline
  0100 0010 &	102 &	66 &	42 &	B\\\hline
  0100 0011 &	103 &	67 &	43 &	C\\\hline
  0100 0100 &	104 &	68 &	44 &	D\\\hline
  0100 0101 &	105 &	69 &	45 &	E\\\hline
  0100 0110 &	106 &	70 &	46 &	F\\\hline
  0100 0111 &	107 &	71 &	47 &	G\\\hline
  0100 1000 &	110 &	72 &	48 &	H\\\hline
  0100 1001 &	111 &	73 &	49 &	I\\\hline
\end{tabular}
\end{column}
\begin{column}{.5\textwidth}
  \begin{tabular}{|c|c|c|c|c|}\hline
 \bf bin &	\bf oct &\bf	dec &\bf	hex &\bf	letra\\\hline
   0110 0001 &	141 &	97 &	61 &	a \\\hline
    0110 0010 &	142 &	98 &	62 &	b \\\hline
    0110 0011 &	143 &	99 &	63 &	c \\\hline
    0110 0100 &	144 &	100 &	64 &	d \\\hline
    0110 0101 &	145 &	101 &	65 &	e \\\hline
    0110 0110 &	146 &	102 &	66 &	f \\\hline
    0110 0111 &	147 &	103 &	67 &	g \\\hline
    0110 1000 &	150 &	104 &	68 &	h \\\hline
    0110 1001 &	151 &	105 &	69 &	i \\\hline
  \end{tabular}
\end{column}
\end{columns}
\end{frame}

\begin{frame}{Dígitos}{ASCII}

  Representação dos dígitos:\bigskip
  
\begin{center}
  \begin{tabular}{|c|c|c|c|c|}\hline
    \bf bin &	\bf oct &\bf	dec &\bf	hex &\bf dígito\\\hline
    0011 0001 &	061 &	49 &	31 &	1  \\\hline
    0011 0010 &	062 &	50 &	32 &	2 \\\hline
    0011 0011 &	063 &	51 &	33 &	3 \\\hline
    0011 0100 &	064 &	52 &	34 &	4 \\\hline
    0011 0101 &	065 &	53 &	35 &	5 \\\hline
    0011 0110 &	066 &	54 &	36 &	6 \\\hline
    0011 0111 &	067 &	55 &	37 &	7 \\\hline
    0011 1000 &	070 &	56 &	38 &	8 \\\hline
    0011 1001 &	071 &	57 &	39 &	9 \\\hline
  \end{tabular}
\end{center}

\end{frame}

\begin{frame}{Códigos de controle}{ASCII}
  
  Alguns códigos de controle:\bigskip
  
  \begin{tabular}{|c|c|c|c|c|c|}\hline
    \bf bin &	\bf oct &\bf	dec &\bf	hex &\bf código	em C & \bf função \\\hline
    0000 0000 &	000 &	00 &	00 	& 	 	$\backslash$0    &	nulo  \\\hline
    0000 1001 &	011 &	09 &	09 	& 	$\backslash$t & tabulação horizontal \\\hline
    0000 1010 &	012 &	10 &	0A 	&	$\backslash$n & nova linha \\\hline
    0001 1011 &	033 &	27 &	1B 	& 	              & escapar (Esc)  \\\hline
    0111 1111 &	177 &	127 &	7F 	& 	 	      & apagar (Del) \\\hline
  \end{tabular}

\end{frame}

\subsection{Padrão Unicode}

\begin{frame}{Padrão Unicode}
  
  O Padrão \alert{\href{http://www.unicode.org/}{Unicode}} é usado
  para representação de texto no computador, englobando a maioria das
  línguas existentes.

  O padrão ISO/IEC~10646 suporta 3 formatos de codificação:

  \begin{description}
  \item[UTF-8:] Os caracteres possuem tamanho variável e correspondência com o ASCII.
  \item[UTF-16] Representa a maioria dos caracteres com 16 bits.
  \item[UTF-32] Possui tamanho fixo de 32 bits, usado quando não houver preocupação com 
    armazenamento.
  \end{description}
  
  Todas as codificações usam 4 bytes no máximo.

\end{frame}

\begin{frame}{Codificação UTF-8}{Unicode}
  
  A codificação UTF-8 foi desenvolvida por Ken Thompsom e Rob Pike em
  1992, com o objetivo de ser compatível com o padrão ASCII e evitar
  problemas de portabilidade devido à representação de bits.\medskip

  O padrão é recomendado pelo consórcio W3C e {\em Internet Mail} para
  representação dos caracteres.\medskip
  
  Exemplo: {\em tag} HTML indicando o uso da codificação UTF-8\medskip 
 
  {\tt <meta http-equiv="Content-Type" content="text/html; charset=utf-8" />}
  

\end{frame}

\begin{frame}{Codificação UTF-8}{Unicode}
  
  Boa parte dos sistemas operacionais e linguagens de programação
  atuais utilizam UTF-8 para codificação dos caracteres.\medskip
  
  O número de bytes utilizados varia de acordo com o conjunto de
  caracteres a serem representados:

  \begin{description}
  \item[1 byte] Caracteres ASCII.
  \item[2 bytes] Alfabetos latino, grego, cirílico, arábico, hebreu, armênio.
  \item[3 bytes] Restante das línguas.
  \item[4 bytes] Símbolos matemáticos, símbolos pictóricos.
  \end{description}

\end{frame}
