\lecture{Entrada/Saída e Armazenamento}{storage}

\lecturetitle{\insertlecture}{\course}

\section{\insertlecture}

\frame{\maketitle}

\note{A Figura~\ref{fig:iointerconnect} mostra a visão abstrata de um
sistema simples com sua estrutura de entrada-saída (E/S). A
transmissão de dados pela interconexão de memória-E/S forma uma rede
complexa devido à diversidade de dispositivos. Por este motivo
detalhes da comunicação no barramento foram ocultados.}

\begin{frame}{Interconexão de E/S}
\begin{figure}[h]
  \centering
  
%%% Local Variables: 
%%% mode: latex
%%% TeX-master: "../notes"
%%% End: 

\def\intershift{1.2cm}
\begin{tikzpicture}
  \tikzset{scale=.65,
    every node/.style={font=\scriptsize,text centered},
    every path/.style={draw},
    disk/.style={cylinder,shape border rotate=90,aspect=0.5,
      minimum width=1cm,minimum height=1.75cm,draw},
  dev/.style={text width=1.5cm,anchor=west,draw}}

  \node[dev] (proc) at (0,0) {Processador};
  \node[dev,text width=1cm] (cache) [below of=proc] {Cache};
  \node[minimum width=7*\intershift,anchor=west,draw] (bus)  [below of=cache] {Interconexão de memória e entrada e saída};
  \node[dev] (mem) [below of=bus,xshift=-1.5*\intershift,xshift=-2*\intershift] {Memória principal};
  \node[dev] (diskcon) [right of=mem,xshift=1.5*\intershift] {Controlador de E/S};
  \node[dev] (videocon) [right of=diskcon,xshift=1.5*\intershift] {Controlador de E/S};
  \node[dev] (netcon) [right of=videocon,xshift=1.5*\intershift] {Controlador de E/S};
  % dispositivos
  \node[disk] (disk1) [below of=diskcon,xshift=-.55*\intershift,yshift=-\intershift] {Disco};
  \node[disk] (disk2) [below of=diskcon,xshift=.55*\intershift,yshift=-\intershift] {Disco};
  \node[draw] (video) [below of=videocon] {Saída gráfica};
  \node[tape,minimum width=2cm,draw] (net) [below of=netcon] {Rede};
  % paths
  %% bus
  \path (proc) -- (cache);
  \path (cache) -- (bus);
  \path (mem) -- (bus.south west);
  \path (diskcon) -- (bus);
  \path (videocon) -- (bus);
  \path (netcon) -- (bus.south east);
  %% disks
  \path (diskcon) -- (disk1);
  \path (diskcon) -- (disk2);
  %% video
  \draw (videocon) -- (video);
  %% net
  \draw (netcon) -- (net);

\end{tikzpicture}
  \caption{Visão abstrata de um sistema com sua estrutura E/S.}
  \label{fig:iointerconnect}
\end{figure}
\end{frame}

\begin{frame}{Performance}
  Para determinação da performance do dispositivo é preciso determinar a
  largura de banda ({\em bandwidth}) que pode ser obtida por este
  dispositivo no barramento. Há dois tipos de medidas que podem ser
  realizadas para determinar a largura de banda de E/S:

  \begin{enumerate}
  \item Quantidade de dados tranportados em determinado período de tempo;
  \item Número de operações de entrada e saída que podem ser realizadas
    em um determinado período de tempo.
  \end{enumerate}

\end{frame}

% \begin{frame}{Especificação de serviços de armazenamento}
% \begin{figure}[h]
% \begin{center}
% 
%%% Local Variables: 
%%% mode: latex
%%% TeX-master: "../notes"
%%% End: 


\begin{tikzpicture}
  \small
  \tikzset{->,>=latex,
    el/.style={above} % edge label
  }

  \node[state] (spec) {Especificado};
  \node[state,draw=red] (nospec) [right of=spec,xshift=50mm] {\color{red} Não Especificado};
  
  \path[color=red] (spec) edge [bend left] node[el] {\color{red} Falha}  (nospec);
  \path (nospec)  edge  node[el] {Recuperação} (spec);
\end{tikzpicture}
% \end{center}
% \label{state:spec}
% \caption{Diagrama de transição de estados relacionado a falha e
%   recuperação de um disco.}
% \end{figure}
% \end{frame}

% \begin{frame}{Conceitos básicos de armazenamento}
% \begin{description}
% \item[Confiabilidade ($t^C$):] é a medida de fornecimento contínuo do serviço
%   pelo sistema conforme especificado;
% \item[Recuperação ($t^R$):] é a medida quantitativa de tempo que o sistema
%   levou para sair do estado não especificado para o especificado, ou
%   seja, é o tempo de reparo;
% \item[Disponibilidade ($t^D$):] é a medida quantitativa do tempo em que sistema
%   forneceu o serviço conforme especificado;
%   \begin{equation}
%     \label{eq:avail}
%     t^D = {\sum_{i=1}^{N}{t_i^C} \over {\sum_{i=1}^{N}{t_i^C} + \sum_{i=1}^{N}{t_i^R}}}
%   \end{equation}
% \noindent onde $N$ é número de transições do estado {\em especificado} para
% o estado {\em não especificado}, ou seja, o número de falhas.
% \end{description}
% \end{frame}


\subsection{Armazenamento em disco}

\begin{frame}{Disco Rígido}
O disco rígido (Figura~\ref{fig:hd}) é composto por bandejas rotativas
com superfície magnética e cabeçote deslizante de acesso ao disco para
leitura e escrita.

  \begin{figure}[h]
  \centering


%\usepackage{tikz}


\begin{center}
  \begin{tikzpicture}[pointtrack/.style={->,color=blue}]
   \node[fill=cyan,circle] at (0,0) {\bf eixo};
   \foreach \c in {0,...,8} {
     \draw[thick=2pt] (0,0) circle  (12*\c pt);
     \draw[loosely dotted,color=white,dotted,thick=3pt] (0,0) circle  (12*\c pt);
   }
   \node[color=blue] (track) at (4,-2) {trilhas};
   \foreach \x in {3,2,1} {
      \draw[pointtrack] (track) -- (\x,-1);
    }
    \node[circle,rounded corners=3mm,very thick,red,draw] (focus) at
    (-3.2,-1)  {};
    \node (sector) at (-5,-1) {setores};
    \draw[->,red]  (sector) -- (focus);
  \end{tikzpicture}
\end{center}


\label{fig:hd}
\caption{Esquema de disco rígido.}
\end{figure}
\end{frame}

\begin{frame}{Característica de um disco rígido}
\begin{itemize}
\item Não volátil: mesmo sem energia, conserva o estado dos dados armazenados;
\item Bandejas: 1 -- 4;
\item Velocidade de rotação: 5.400 -- 15.000 RPM;
\item Diâmetro da bandeja: 1 -- 3,5 polegadas;
\item Trilhas: 10.000 -- 50.000 por superfície;
\item Setor: 512 B.
\end{itemize}
\end{frame}


\begin{frame}{Estágios de acesso ao disco rígido}
\small
  O termo {\bf cilindro} refere-se às trilhas sob o cabeçote em cada
superfície.

Para acessar os dados o sistema operacional deve acessar diretamente o
disco em um processo de 3 estágios:


  \begin{enumerate}
\item {\bf Busca da trilha:} o cabeçote procura a trilha desejada;
\item {\bf Latência de rotação:} o cabeçote procura o setor desejado;
  $$\overline{t}^R = {0,5\ R \over 5.400\ RPM}= { 0,5 R \over {5.400
    R/min \over (60 s/min)}} = 0,0056\ s = 5,6\ ms$$
  $$\overline{t}^R = {0,5\ R \over 15.000\ RPM}= { 0,5 R \over {15.000
    R/min \over (60 s/min)}} = 0,0020\ s = 2,0\ ms$$

Calcular o tempo médio de rotação para acesso aos dados de um disco
com velocidade de rotação de 7.200 RPM.

\item {\bf Tempo de transferência:} tempo para transferir um bloco de bits.
\end{enumerate}
\end{frame}

\begin{frame}{Exercício 1.} 
Qual o tempo médio para ler ou escrever um setor de 512 bytes para um
disco com rotação de 15.000 RPM? O tempo médio de busca é 4 ms, o
tempo de transferência é 100 MB/s, e o tempo de atraso da controladora
é 0,2 ms. Assumir que o disco está ocioso, ou seja, não há tempo de
espera.

\pause

$$ 4,0 ms + {0,5\ R \over 15.000\ RPM} + {0,5\ KB \over 100\ MB/s} +
0,2\ ms = 6,2\ ms$$

\pause\bigskip\small\color{blue}
Para os discos rígidos atuais, a interface controladora
convergiu para dois padrões principais:
\begin{itemize}
\item SATA -- {\em Serial Advanced Technology Attachment};
\item SCSI -- {\em Small Computer System Interface}.
\end{itemize}

\end{frame}


\subsection{Memória flash}

\begin{frame}{Características}
\begin{itemize}
\item Semi-condutora;
\item Latência de 100 a 1.000 menor quando comparada ao disco;
\item Ocupa menos espaço;
\item Mais eficiente;
\item Resistente a queda;
\item Popular em dispositivos móveis (devido ao tamanho e pouca escrita);
\item EEPROM.
\end{itemize}

Tempo de escrita = 1.500 ms para apagar + 250 ms. Por este motivo não
é popular em servidores e desktops.

\end{frame}

\begin{frame}{Exercício 2} 

  Calcular o tempo médio de {\bf leitura} em uma memória flash com blocos de
  1024 bytes com taxa de transferência de dados de 32 MB/s e taxa de
  transferência do controlador de 360 MB/s.

\pause

$${1\ K \over 32\ MB/s}+{1\ K \over 360 MB/s} = 0,034\ ms$$

\end{frame}

\bibliographystyle{plain}
\bibliography{cod}

\end{document}
