
\def\myfigure{{\tt\vspace{3cm}\Large FIGURA\vspace{3cm}}}

\section{Circuitos Sequenciais}

\subsection*{Breve revisão}

\paragraph{Clocks.}~A abordagem com uso de clocks permite determinar
quando o dado é válido e estável. O sinal do clock oscila entre
valores de alta e baixa voltagem.


\myfigure

\paragraph{Elemento de estado.}~É o elemento de armazenamento ou
memória.

\myfigure

\paragraph{Sistema síncrono.}~Sistema de memória que emprega clocks e
onde os sinais de dados são lidos somente quando o clock indica que o
valor é estável.

\paragraph{Latch.}~Elemento de memória em que a saída é igual ao valor
de estado de armazenamento dentro do elemento e o estado é alterado
sempre que as entradas apropriadas alteram-se e o clock está ativado.


\myfigure


\paragraph{Flip-flop.}~Elemento de memória em que a saída é igual ao
valor do estado interno armazenado e o estado é alterado somente na
borda do clock.

\myfigure


\paragraph{Circuito combinacional.} Circuito sem memória em que a
saída depende somente da entrada.

\myfigure


\noindent Exemplos:

\noindent Multiplexador 2-para-1

\myfigure

\noindent Decodificador 2-para-4


\myfigure


\subsection*{Análise de Circuitos Sequenciais}


\noindent Latch S-R com porta NOR

\myfigure

\noindent Tabela de estados

\begin{center}
\begin{tabular}[ht]{cccc}\hline
  & & atual & próximo \\\hline
  S=$x_1$ & R=$x_2$ & Q=$f$ & $f^+$\\\hline
  0& 0 & 0 & 0\\
  0& 0 & 1 & 1\\
  0& 1 & 0 & 0\\
  0& 1 & 1 & 0\\
  1& 0 & 0 & 1\\
  1& 0 & 1 & 1\\
  1& 1 & 0 & Inválido\\  
  1& 1 & 1 & Inválido\\
\end{tabular}
\end{center}

Como analisar o circuito sequencial que é baseado em flip-flops que
são construídos com latches?


\myfigure

\subsubsection{Máquina de Estado Finito}

Tabelas de estado e diagramas de tempo descrevem o comportamento dos
circuitos sequenciais. Uma descrição gráfica equivalente é fornecida
por uma {\bf Máquina de Estado Finito (MFE)}. As MFEs usam círculos
para representar os estados e arcos direcionados para representar as
transições de um estado a outro.

\noindent Exemplo: máquina de estados de um semáforo

\myfigure


Das máquinas de estado finito criadas para analisar circuitos
sequenciais destacam-se:

\begin{itemize}
\item Máquina de Moore;
\item Máquina de Mealy.
\end{itemize}


\subsubsubsection{Máquina de Moore}

Criada por Edward F.\ Moore em 1956, é uma MEF com as seguintes
propriedades:

\begin{itemize}
\item O próximo estado depende somente do estado atual;
\item As saída alteram na borda do clock.
\end{itemize}


\myfigure


\subsubsubsection{Máquina de Mealy}

Criada em 1955 por George H.\ Mealy, é uma MEF com as seguintes
propriedades:

\begin{itemize}
\item O próximo estado depende do estado atual e das entradas;
\item Reagem no mesmo ciclo, ou seja, não esperam pela borda do clock.
\end{itemize}

\myfigure


\subsubsection{Análise de Circuitos Sequenciais usando MEFs}


\paragraph{Máquina de Moore do Latch SR.} As máquinas de Moore e Mealy
podem ser descritas de forma completa pela quíntupla $M=
\<Q,S,\Sigma,\Gamma,\delta> $, onde:

\begin{description}
  \item[$Q$] -- conjunto finito que representa as configurações da máquina;
  \item[$S$] -- configuração inicial antes de receber as entradas;
  \item[$\Sigma$] -- alfabeta de entrada ou conjunto de eventos que a
    máquina reconhece;
  \item[$\Gamma$] -- alfabeto finito de saída;
  \item[$\sigma$] -- função Booleana de transição que mapeia o próximo
    estado.

\end{description}

\noindent Exemplo: Latch S-R

\begin{figure}[ht]
\centering
\begin{tikzpicture}[state/.style=state with output]
  \def\D{2cm}
  \node[state] (q0) {$q_0$ \nodepart{lower} $[0]$};
  \node[state] (q1) [right of=q0,xshift=\D] {$q_1$ \nodepart{lower} $[1]$};
  \node[state] (q2) [right of=q0,xshift=\D/2.75,yshift=-\D] {$\epsilon$};
  
  \path[->] (q0) edge[bend left] node[above] {10} (q1)
  (q0) edge[bend right] node[left] {11} (q2)
  (q1) edge[bend left] node[above] {01} (q0)
  (q1) edge[bend left] node[right] {11} (q2)
  (q0) edge[loop above] node {00,01} ()
  (q1) edge[loop above] node {00,10} ()
  ;
\end{tikzpicture}
\caption{Estados da máquina de Moore para o latch SR.}
\label{fig:mooreSR}
\end{figure}

A descrição formal da máquina de Moore que representa o latch S-R é
$Mo=\<Q,\Sigma,\delta,S,\Gamma>$ onde:

\begin{enumerate}
\item $\epsilon=$ estado ``Inválido''
\item $Q = \{q_0, q_1,\epsilon\}$
\item $\Sigma = \{0,1\}$
\item $\delta=$\\

  \begin{center}
  \begin{tabular}[ht]{c|cccc}
    & $00$ & $01$ & $10$ & $11$ \\\hline
    $q_0[0]$ & $\set{q_0}$ & $\set{q_0}$ & $\set{q_1}$ & $\set{\epsilon}$\\
    $q_1[1]$ & $\set{q_1}$ & $\set{q_0}$ & $\set{q_1}$ & $\set{\epsilon}$\\
  \end{tabular}
\end{center}

\item $S=q : q \in \{q_0, q_1\}$ 

\item $\Gamma=q : q\in \{q_0,q_1\}$ 

\end{enumerate}

\paragraph{Máquina de Mealy do Latch SR.}

\begin{figure}[ht]
\centering
\begin{tikzpicture}[]
  \def\D{2cm}
  \node[state] (q0) {$q_0$};
  \node[state] (q1) [right of=q0,xshift=\D] {$q_1$};
  \node[state] (q2) [right of=q0,xshift=\D/2.75,yshift=-\D] {$\epsilon$};
  
  \path[->] (q0) edge[bend left] node[above] {10/0} (q1)
  (q0) edge[bend right] node[left] {11/0} (q2)
  (q1) edge[bend left] node[above] {01/1} (q0)
  (q1) edge[bend left] node[right] {11/1} (q2)
  (q0) edge[loop above] node {00/0,01/0} ()
  (q1) edge[loop above] node {00/1,10/1} ()
  ;
\end{tikzpicture}
\caption{Estados da máquina de Mealy para o latch SR.}
\label{fig:mealySR}
\end{figure}



A descrição formal da máquina de Mealy que representa o latch S-R é
$Me=\<Q,\Sigma,\delta,S,F>$ onde:

\begin{enumerate}
\item $\epsilon=$ estado ``Inválido''
\item $Q = \{q_0, q_1,\epsilon\}$
\item $\Sigma = \{0,1\}$
\item $\delta=$\\

  \begin{center}
  \begin{tabular}[ht]{c|cccccccc}
    & $00/0$ & $00/1$ & $01/0$ & $01/1$ &  $10/0$ & $10/1$ & $11/0$ & $11/1$ \\\hline
    $q_0$ &  $\set{q_0}$ & $\emptyset$ & $\set{q_0}$ &
    $\emptyset$ &$\set{q_1}$ & $\emptyset$& ``Inválido'' & $\emptyset$ \\
    $q_1$ & $\emptyset$  &  $\set{q_1}$& $\emptyset$ &
    $\set{q_0}$ & $\emptyset$ & $\set{q_1}$ &  $\emptyset$&``Inválido''\\
  \end{tabular}
\end{center}

\item $S= q : q \in \{q_0, q_1\}$ 

\item $\Gamma= q : q \in \{q_0, q_1\}$ 

\end{enumerate}

Para as próximas MEFs, iremos construir as máquinas de Moore e Mealy
se forem diferentes, senão, construiremos apenas a máquina de Moore.
