\documentclass{article}

\usepackage[brazilian]{babel}
\usepackage[utf8]{inputenc}
\usepackage[brazilian]{babel}

% additional packages
\usepackage{hyperref}
\usepackage{ifpdf}
\usepackage{tikz}
\usepackage{pgffor}

\pagestyle{empty}

\def\theradius{1.5cm}

% Ideia retiradas do link
% http://www.csunplugged.org/binary-numbers

\begin{document}

\begin{tikzpicture}
  \draw[draw=white] (0,0.1\textwidth) circle (\theradius);
  \draw[fill] (.4\textwidth,0.6\textheight) circle (\theradius);
\end{tikzpicture}
\pagebreak

\begin{tikzpicture}
  \foreach \p in {0.1,0.9} {
  \draw[fill] (\p\textwidth,\p\textheight) circle (\theradius);
}
\end{tikzpicture}
\pagebreak

\begin{tikzpicture}
  \foreach \x/\y in {0.1/0.1,0.9/0.1,0.1/0.9,0.9/0.9} {
  \draw[fill] (\x\textwidth,\y\textheight) circle (\theradius);
}
\end{tikzpicture}
\pagebreak

\begin{tikzpicture}
  \foreach \x in {0.1,0.9}
    \foreach \y in {0.1,0.4,0.65,0.9} 
      \draw[fill] (\x\textwidth,\y\textheight) circle (\theradius);

\end{tikzpicture}
\pagebreak


\begin{tikzpicture}
  \foreach \x in {0,0.3,0.6,0.9}
    \foreach \y in {0,0.3,0.6,0.9} 
      \draw[fill] (\x\textwidth,\y\textheight) circle (\theradius);

\end{tikzpicture}

\pagebreak

\begin{tikzpicture}
  \draw[line width=10pt] (0.7\textwidth,0.5\textheight) ellipse (5cm and 11cm);
\end{tikzpicture}

\end{document}

