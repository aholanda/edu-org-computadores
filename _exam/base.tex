\section*{Conversão de bases}

\begin{minipage}[h]{.5\linewidth}
\noindent 1. Converta os numeros a seguir para a base decimal:
\begin{enumerate}[a)]
\item $101110_2$ % 22
\item $10111001_2$ % 185
\item $243_8$ % 163
\item $2165_8$ % 1141
\item $1FA2_{16}$ % 8096
\item $E1AC_{16}$ % 57772
\end{enumerate}
\end{minipage}
\begin{minipage}[h]{.5\linewidth}
\noindent 2. Converta os numeros a seguir da base decimal para as bases especificadas:
\begin{enumerate}[a)]
\item $23\rightarrow 2$ % 10111
\item $35\rightarrow 2$ % 100011
\item $73\rightarrow 8$ % 111
\item $134\rightarrow 8$ % 205
\item $213\rightarrow 16$ % D5
\item $3315\rightarrow 16$ % CF3
\end{enumerate}
\end{minipage}

\bigskip\noindent\begin{minipage}[h]{.5\linewidth}
\noindent 3. Converta os numeros a seguir para a base binária:
\begin{enumerate}[a)]
\item $23_8$ % 10 011
\item $341_8$ % 11 100 001
\item $1207_8$ % 1 010 000 111
\item $AC_{16}$ % 1010 1100
\item $ACBF_{16}$ % 1010 1100 1011 1111
\item $9CE5B_{16}$ % 1001 1100 1110 0101 1011
\end{enumerate}
\end{minipage}
\begin{minipage}[h]{.5\linewidth}
\noindent 4. Converta os numeros a seguir da base binária para as bases octal e hexadecimal:
\begin{enumerate}[a)]
\item $010_2$ % 2
\item $1010_2$ % 12, A 
\item $10111_2$ % 27, 17
\item $01010111_2$ % 127, 57
\item $10010101_2$ % 225, 95
\item $110100110_2$ % 646, 1A6
\end{enumerate}
\end{minipage}

\bigskip
\noindent 5. Considere os binários do exercício~4 na representação complemento de 2 
com número variável de bits, e converta-os para a base decimal.

\medskip\noindent 6. Converta os números a seguir de decimal para
binário com sinal (complemento de 2), e realize a operação de
inversão, mostrando os cálculos intermediários. O número de bits é
indicado entre parênteses.

\begin{enumerate}[a)]
\item 28 (6 bits)
\item -23 (6 bits)
\item -32 (6 bits)
\item -1 (6 bits)
\item -9 (5 bits)
\item -1 (5 bits)
\item -43 (7 bits)
\item 33 (7 bits)
\end{enumerate}
