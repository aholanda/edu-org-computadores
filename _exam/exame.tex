\documentclass[12pt]{article}

\usepackage[brazil]{babel}
\usepackage[utf8]{inputenc}
\usepackage{a4wide}

\author{Prof. Adriano J. Holanda}
\title{Exame de Introdu\c{c}\~ao \`a Computa\c{c}\~ao II}
\date{\today}

\def\mynot#1{\overline{#1}}

\begin{document}

\maketitle

\noindent Faculdade "Dr. Francisco Maeda" -- FAFRAM\\
\noindent Curso: Sistemas de Informação, $2^o$ ciclo\\
\noindent Nome completo:

\paragraph{Quest\~ao 1. (5,0 pontos)} Converta os n\'umeros a seguir de
acordo com as bases indicadas.

\begin{enumerate}
\item bin\'ario $\rightarrow$ decimal
  \begin{enumerate}
  \item 10110
  \item 111011
  \item 10010101
  \end{enumerate}
\item decimal $\rightarrow$ bin\'ario
  \begin{enumerate}
    \item 25
    \item 52
  \end{enumerate}
\item decimal $\rightarrow$ hexadecimal
    \begin{enumerate}
      \item 59
      \item 53
    \end{enumerate}
  \item hexadecimal $\rightarrow$ decimal
  \begin{enumerate}
    \item 743
    \item 58
    \item 72
  \end{enumerate}
\end{enumerate}

\paragraph{Quest\~ao 2. (5,0 pontos)}
Simplifique as expressões {\em booleanas} a
seguir:

\begin{enumerate}
\item (1,5 ponto) $y = A\mynot{B}D + \mynot{AB}\ \mynot{D}$ % Resposta: \mynot{AB}
\item (1,5 ponto) $x = ACD + \mynot{A}BCD$ % R: ACD + BCD
\item (2 pontos) $z = (\mynot{A} + B)(A+B)$ % R: B
\end{enumerate}

%\bigskip
%\subsubsection*{Teoremas {\em booleanos}}

%\scriptsize
%\input{../boolean_theorems}


\end{document}
