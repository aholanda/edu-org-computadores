\maketitle

\section*{Circuitos lógicos combinacionais}

\noindent Fonte:
\href{https://feituverava.bv3.digitalpages.com.br/users/publications/9788576059226/pages/_1}{Ronald
  J. Tocci; Neal S. Widmer; Gregory L. Moss. Sistemas Digitais -
  Princípios e Aplicações, 11$^{a.}$ edição. Pearson, 2011.}

\exercise~Para cada uma das expressões a seguir, desenhe o circuito
lógico correspondente usando portas AND, OR e NOT.

\begin{minipage}{.55\textwidth}
  \begin{enumerate}[(a),series=mylist]
  \item $x = \NOT{AB(C+D)}$
  \item $y = (\NOT{M+N}+\NOT{P}Q)$
  \item $x=\NOT{W+P\NOT{Q}}$
  \item $z=MN(P+\NOT{N})$
  \item $x=(A+B)(\NOT{A}+\NOT{B})$
  \item $x=AB(C+\NOT{B})$
  \end{enumerate}
\end{minipage}
\begin{minipage}{.4\textwidth}
  \begin{enumerate}[resume*=mylist]
  \item $y=\NOT{A}\NOT{B}\NOT{C}$
  \item $y=\NOT{A}D+BD$
  \item $y=(\NOT{R}+S+\NOT{T})Q$
  \item $y=\NOT{A+B+\NOT{C}D}$
  \item $y=\NOT{\NOT{A}+\NOT{C}+\NOT{D}}$
  \item $y=\NOT{A(\NOT{B+\NOT{C}})D}$
  \end{enumerate}
\end{minipage}


\exercise Analise os circuitos das Figuras~\ref{fig:fourgates} e
\ref{fig:circ} e descreva-os utilizando a álgebra booleana. Para quais
entradas (valores) de $A$, $B$, $C$, respectivamente, dos circuitos
das Figuras~\ref{fig:fourgates} e \ref{fig:circ}, a saída
será $1$? (Dica: Use a tabela-verdade para a análise.)\\

\begin{figure}[ht]
\begin{center}
\begin{tikzpicture}
  \node[name=a,draw, and gate US]  at (0,3.25) {};
  \node[name=A] at (-2,3.35) {\tiny A};
  \draw[] (A) -- (a.input 1);

  \node[name=b,draw, or gate US]  at (0,2.25) {};
  \node[name=B] at (-2,3.175) {\tiny B};
  \draw[] (B) -- (a.input 2);
  \node[name=C] at (-2,2.175) {\tiny C};
  \draw[] (C) -- (b.input 2);
  \draw (b.input 1) -| (-1,3.175) node[] {};

  \node[name=c,draw, and gate US] at (0,1.25) {};
  \draw (c.input 1) -| (-1,2.175) {};
  \draw (c.input 2) -| (-1.4,3.175) node[circle,draw,dashed] (1pt) {};


  \node[name=d,draw, or gate US] at (3,3.15) {};
  \node[name=C] at (4.5,3.15) {\scriptsize Q};
  \draw[] (d.output) -- (C);

  \node[name=e,draw, and gate US] at (1.5,1.75) {};
  \draw[] (a.output) -- (d.input 1);
  \draw[] (b.output) -| (1,1.825) |- (e.input 1);
  \draw[] (c.output) -| (1,1.25) |- (e.input 2);
  \draw[] (e.output) -| (2.5,3) |- (d.input 2);

  \end{tikzpicture}
\end{center}
\caption{Circuito lógico combinacional 1}
\label{fig:fourgates}
\end{figure}

% Resposta: $AB+BC(B+C)$ 

\begin{figure}[ht]
\begin{center}
\begin{tikzpicture}
  \node[name=a,draw, nor gate US]  at (0,3.25) {};
  \node[name=A] at (-2,3.35) {\tiny A};
  \draw[] (A) -- (a.input 1);

  \node[name=b,draw, nand gate US]  at (0,2.25) {};
  \node[name=B] at (-2,3.175) {\tiny B};
  \draw[] (B) -- (a.input 2);
  \node[name=C] at (-2,2.175) {\tiny C};
  \node[name=n,draw, not gate US]  at (-1.3,2.175) {};
  \draw (C) -- (n.input);
  \draw (n.output) -- (b.input 2);
  \draw (b.input 1) -| (-1,3.175) node[circle] {};


  \node[name=c,draw, nor gate US] at (3,3.15) {};
  \node[name=X] at (4.5,3.15) {\scriptsize X};
  \draw (a.output) -- (c.input 1);
  \draw (c.output) -- (X);
  \draw (b.output) -| (1,2.75) |- (c.input 2) {};

  \end{tikzpicture}
\end{center}
\caption{Circuito lógico combinacional 2.}
\label{fig:circ}
\end{figure}

% Resolução:  0,1,0 

% \bigskip
% \begin{center}
% \begin{tabular}{|c|c|c|c|c|c|c|}
% $A$ & $B$ & $C$ &$y=\NOT{A+B}$ &$\notc$ & $z=\NOT{B\notc}$
% &$x=\NOT{y+z}$ \\\hline
% 0 & 0 & 0 & 1 & 1 & 1 & 0\\
% 0 & 0 & 1 & 1 & 0 & 1 & 0\\
% 0 & 1 & 1 & 0 & 0 & 1 & 0\\
% 1 & 1 & 1 & 0 & 0 & 1 & 0\\\hline
% 1 & 1 & 0 & 0 & 1 & 0 & 1\\\hline
% 1 & 0 & 0 & 0 & 1 & 1 & 0\\
% 1 & 0 & 1 & 0 & 0 & 1 & 0\\\hline
% \bf 0 & \bf 1 & \bf 0 &\bf 0 &\bf 1 &\bf 0 &\bf 1\\\hline

% \end{tabular}
% \end{center}

