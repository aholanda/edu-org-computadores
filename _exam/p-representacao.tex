\question{1,5} Converta os números a seguir para base binária:

\begin{enumerate}[a)]
\item $43_8$
\item $436_8$
\item $33_{10}$
\item $44_{10}$
\item $A48_{16}$
\item $5BCD3_{16}$
\end{enumerate}

\question{1} Converta os números a seguir de decimal para
binário com sinal (complemento de 2). O número de bits é
indicado entre parênteses.

\begin{enumerate}[a)]
\item 26 (6 bits)
\item -21 (6 bits)
\item -31 (6 bits)
\item -33 (7 bits)
\end{enumerate}

\question{1,5}~Realize as seguintes operações aritméticas (não é
necessário realizar a conversão de bases, porém, mostre as operações
intermediárias):\bigskip

\begin{minipage}{.5\textwidth}
\begin{center}
  \begin{tabular}[h]{rc}
     & $0\ 1\ 0\ 1\ 1$ \\
    $+$& $0\ 1\ 0\ 0\ 1$ \\\hline
  \end{tabular}
\end{center}
\end{minipage}
\begin{minipage}{.5\textwidth}
\begin{center}
  \begin{tabular}[h]{rc}
     & $1\ 0\ 0\ 1\ 0\ 1\ 0$ \\
    $+$& $0\ 1\ 0\ 1\ 1\ 0\ 1$ \\\hline
  \end{tabular}
\end{center}
\end{minipage}\vspace{1.5cm}

\begin{minipage}{.5\textwidth}
\begin{center}
  \begin{tabular}[h]{rc}
        &$1\ 0\ 0\ 1\ 0$ \\
    $-$ &$0\ 0\ 1\ 0\ 1$ \\\hline
  \end{tabular}
\end{center}
\end{minipage}
\begin{minipage}{.5\textwidth}
\begin{center}
  \begin{tabular}[h]{rc}
        &$1\ 0\ 0\ 0\ 0\ 0\ 1$ \\
    $-$ &$0\ 0\ 0\ 0\ 1\ 1\ 1$ \\\hline
  \end{tabular}
\end{center}
\end{minipage}\vspace{1.5cm}

\begin{minipage}{.5\textwidth}
\begin{center}
  \begin{tabular}[h]{rc}
            & $1\ 0\ 1\ 1$ \\
    $\times$& $0\ 0\ 1\ 1$ \\\hline
  \end{tabular}
\end{center}
\end{minipage}
\begin{minipage}{.5\textwidth}
\begin{center}
  \begin{tabular}[h]{rc}
     & $0\ 1\ 1\ 0\ 1\ 0$ \\
    $\times$& $0\ 0\ 0\ 1\ 0\ 1$ \\\hline
  \end{tabular}
\end{center}
\end{minipage}\vspace{3.5cm}

\question{2} Faça a tabela-verdade para as espressões booleanas a
seguir, e indique para quais entradas a saída será $1$:

\begin{enumerate}[a)]
\item $\~AB+\~{B}$
\item $\~{A}\ + A\~{B}$
\end{enumerate}

\question{3} Para cada uma das expressões a seguir, desenhe o circuito
lógico correspondente usando portas AND, OR e NOT.

\begin{enumerate}[(a)]
\item $f = \~{\~{A}.B(C+D)}$
\item $f=(\~{A+B})(\~{A}+\~{B})$
\item $f=({A.B})({A}.\~{B})$
\end{enumerate}

\question{1} Ache a expressão Booleana do circuito lógico combinacional 
da Figura~\ref{fig:circ}.

\begin{figure}[ht]
\begin{center}
\begin{tikzpicture}
  \node[name=a,draw, or gate US]  at (0,3.25) {};
  \node[name=A] at (-2,3.35) {\tiny A};
  \draw[] (A) -- (a.input 1);

  \node[name=b,draw, and gate US]  at (0,2.25) {};
  \node[name=B] at (-2,3.175) {\tiny B};
  \draw[] (B) -- (a.input 2);
  \node[name=C] at (-2,2.175) {\tiny C};
  \node[name=n,draw, not gate US]  at (-1.3,2.175) {};
  \draw (C) -- (n.input);
  \draw (n.output) -- (b.input 2);
  \draw (b.input 1) -| (-1,3.175) node[circle] {};


  \node[name=c,draw, or gate US] at (3,3.15) {};
  \node[name=X] at (4.5,3.15) {\scriptsize X};
  \draw (a.output) -- (c.input 1);
  \draw (c.output) -- (X);
  \draw (b.output) -| (1,2.75) |- (c.input 2) {};

  \end{tikzpicture}
\end{center}
\caption{Circuito lógico combinacional 1.}
\label{fig:circ}
\end{figure}

\end{document}


\question{2} Prove que: (fazendo a tabela-verdade)

 $$x+(x.y) \equiv x$$


\question{5}~A tabela de cores da linguagem HTML ({\em HyperText
  Markup Language}) utiliza um hexadecimal de 6 dígitos, mapeando cada
2 dígitos para cada intensidade de vermelho, verde e azul (RGB, {\em red,
  green, blue}). Por exemplo, o valor hexadecimal \#DA70D6 representa
a cor orquídea, e seu mapeamento RGB em decimal é (218,112,214), onde
218~(\#DA) é a intensidade de vermelho, 112~(\#70) de verde e
214~(\#D6) de azul. A cerquilha \# indica que o valor é
hexadecimal. Converta os valores de cor a seguir para hexadecimal ou
decimal, conforme o caso:

\begin{enumerate}[a)]
\item Púrpura, \#800080  % (128,0,128)
\item Pink, \#FFC0CB % (255,192,203)
\item Laranja, \#FFA500	% (255,165,0)
%\item Azul, \#0000FF %	(0,0,255)
%\item Sienna, \#A0522D % (160,82,45)
\item Peru, (205,133,63) % \#CD853F
\item Marrom, (165,42,42) % \#A52A2A	
\end{enumerate}


\question{2}~Os valores hexadecimais da questão~1 são convertidos para
a representação binária ao serem armazenados no computador. Converta
as cores Pink e Laranja para a representação binária.

% \begin{minipage}{.5\textwidth}
% \begin{center}
%   \begin{tabular}[h]{rc}
%     $\neg$ & $1\ 0\ 1\ 1\ 0\ 1\ 1$ \\\hline
%   \end{tabular}
% \end{center}
% \end{minipage}
% \begin{minipage}{.5\textwidth}
% \begin{center}
%   \begin{tabular}[h]{rc}
%     $\neg$ & $1\ 0\ 1\ 0\ 1\ 1\ 1\ 1\ 1\ 0\ 0\ 1$ \\\hline
%   \end{tabular}
% \end{center}
% \end{minipage}

