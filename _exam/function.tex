%%
% Local variables:
% mode: latex
% End:
%%

\setcounter{exno}{0}

\section*{Funções Booleanas}

\exercise~Faça a tabela-verdade para as funções Booleanas 
definidas por:

\begin{enumerate}[a)]
\item $f(x_1,x_2,x_3) = x_1(\~x_2 \+ x_3)$
\item $f(a,b,c) = a\~c\+bc\+\~ac$
\end{enumerate}

\exercise~Ache os mintermos, maxtermos das seguintes funções
Booleanas:

\begin{enumerate}[a)]
\item $f(a,b,c) = ab\+ \~c$
\item $f(x_1, x_2, x_3) = (\~x_1 \+ x_2) (\~x_3 \+ x_2)$
\end{enumerate}


\exercise~Ache a forma normal disjuntiva (FND) e forma normal
conjuntiva (FNC) das funções Booleanas para as seguintes
tabelas-verdade:

\begin{enumerate}[a)]
\begin{minipage}{.5\textwidth}
\item $1010$
\item $1101$
\item $1000\ 1001$
\end{minipage}
\begin{minipage}{.5\textwidth}
\item $0010\ 1001$
\item $1000\ 1001\ 0010\ 1001$
\item $0001\ 1100\ 0100\ 1010$
\end{minipage}
\end{enumerate}

\exercise~Construa a tabela contendo somente os mintermos das
seguintes funções Booleanas na FND:

\begin{enumerate}[a)]
\item $f(x,y)=xy\+\~xy$
\item $f(a,b,c)=abc\+ab\~c\+a\~b\~c\+\~a\~bc\+\~a\~b\~c$
\item $f(w,x,y,z)=w\~xyz\+wxy\~z\+wx\~y\ \~z\+\~wx\~yz\+\~wx\~y\ \~z\+\~w\ \~x\ \~y\ \~z$
\end{enumerate}

\exercise~Ache a forma normal disjuntiva, simplifique e desenhe o
circuito lógico combinacional a partir da seguinte tabela-verdade:

\begin{center}
\begin{tabular}{ccc|c}
$a$ & $b$ & $c$ & $f(a,b,c)$\\\hline 
0 & 0 &0  &0\\
0 & 0 &1  &0\\
0 & 1 &0  &0\\
0 & 1 &1  &1\\
1 & 0 &0  &0\\
1 & 0 &1  &1\\
1 & 1 &0  &1\\
1 & 1 &1  &1\\
\end{tabular}
\end{center}

\exercise~Desenhe o circuito comutador mais simples que pode ser
representado pela seguinte tabela:

\begin{center}
\begin{tabular}[h]{cccc|c}
  $p$ & $q$ & $r$ & $s$ & saída \\\hline
  1 & 1 & 1 & 1 & 1 \\
  0 & 1 & 1 & 1 & 1 \\
  1 & 1 & 0 & 1 & 1 \\
  0 & 1 & 0 & 1 & 1 \\
\end{tabular}
\end{center}


\pagebreak
\footnotesize
\section*{Solução}
\setcounter{exno}{0}

\exercise\bigskip

\begin{minipage}[ht]{.5\linewidth}
  a)\\
  \begin{tabular}[ht]{ccc|c}
    $x_1$ & $x_2$ & $x_3$ & $f(x_1,x_2,x_3)$\\\hline
    0 & 0 & 0 & 0\\
    0 & 0 & 1 & 0\\
    0 & 1 & 0 & 0\\
    0 & 1 & 1 & 0\\
    1 & 0 & 0 & 0\\
    1 & 0 & 1 & 1\\
    1 & 1 & 0 & 1\\
    1 & 1 & 1 & 1\\
  \end{tabular}
\end{minipage}
\begin{minipage}[ht]{.5\linewidth}
  b)\\
  \begin{tabular}[ht]{ccc|c}
    $a$ & $b$ & $c$ & $f(a,b,c)$\\\hline
    0 & 0 & 0 & 0\\
    0 & 0 & 1 & 1\\
    0 & 1 & 0 & 0\\
    0 & 1 & 1 & 1\\
    1 & 0 & 0 & 1\\
    1 & 0 & 1 & 0\\
    1 & 1 & 0 & 0\\
    1 & 1 & 1 & 1\\
  \end{tabular}
\end{minipage}

\exercise\bigskip

\begin{minipage}[ht]{.5\linewidth}
  a)\\
  \begin{tabular}[ht]{ccc|c|c|c}
    $a$ & $b$ & $c$ & $f(a,b,c)$ & mintermo & maxtermo\\\hline
    0 & 0 & 0 & 1  & $\!a\,\!b\,\!c$ & \\
    0 & 0 & 1 & 0 & & $a+b+\!c$\\
    0 & 1 & 0 & 1 & $\!ab\!c$ &\\
    0 & 1 & 1 & 0 & & $a+\!b+\!c$\\
    1 & 0 & 0 & 1 & $a\!b\,\!c$&\\
    1 & 0 & 1 & 0 & & $\!a+b+\!c$\\
    1 & 1 & 0 & 1 & $ab\!c$&\\
    1 & 1 & 1 & 1 & $abc$ &\\
  \end{tabular}
\end{minipage}
\begin{minipage}[ht]{.5\linewidth}
  b)\\\def\x{x_1}\def\y{x_2}\def\z{x_3}
  \begin{tabular}[ht]{ccc|c|c|c}
    $\x$ & $\y$ & $\z$ & $f(\x,\y,\z)$ & mintermo & maxtermo\\\hline
    0 & 0 & 0 & 1 & $\!\x\,\!\y\,\!\z$&\\
    0 & 0 & 1 & 0 & & $\x + \y + \!\z$\\
    0 & 1 & 0 & 1 & $\!\x\y\!\z$ &\\
    0 & 1 & 1 & 1 & $\!\x\y\z$ &\\
    1 & 0 & 0 & 0 & & $\!\x+\y+\z$\\
    1 & 0 & 1 & 0 & & $\!\x+\y+\!\z$ \\
    1 & 1 & 0 & 1 & $\x\y\!\z$&\\
    1 & 1 & 1 & 1 & $\x\y\z$ &\\
  \end{tabular}
\end{minipage}

\bigskip\exercise\smallskip

\begin{enumerate}[a)]
\item $f_{FND}(x,y) = \!x\,\!y + x\!y \qquad f_{FNC}(x,y)=(x+\!y)(\!x+\!y)$
\item $f_{FND}(x,y) = \!x\,\!y + \!xy + xy \qquad f_{FNC}(x,y)=\!x+y$

\item $f_{FND}(x,y,z)= \!x\,\!y\,\!z + x\!y\,\!z + xyz$\\
  $f_{FNC}(x,y,z)=(x+y+\!z)(x+\!y+z)(x+\!y+\!z)(\!x+y+\!z)(\!x+\!y+z)$

\item $f_{FND}(x,y,z)=\!xyz + x\!y\,\!z + xyz$\\
  $f_{FNC}(x,y,z)(x+y+z)(x+y+\!z)(x+\!y+\!z)(\!x+y+\!z)(\!x+\!y+z)$

\item $f_{FND}(x,y,z,w) = \!x\,\!y\,\!z\,\!w + \!xy\!z\,\!w + \!xyzw 
  + x\!yz\!w + xy\!z\,\!w + xyzw$\\
  $f_{FNC}(x,y,z,w) = (x+y+z+\!w)(x+y+\!z+w)(x+y+\!z+\!w)(x+\!y+z+\!w)
  (x+\!y+\!z+w)(\!x+y+z+w)(\!x+y+z+\!w)(\!x+y+\!z+\!w)(\!x+\!y+z+\!w)
  (\!x+\!y+\!z+w)$

\item $f_{FND}(x,y,z,w) = \!x\,\!yzw + \!xy\!z\,\!w + \!xy\!zw + x\!y\,\!zw + 
  xy\!z\,\!w + xyz\!w$\\
  $f_{FNC}(x,y,z,w) = (x+y+z+w)(x+y+z+\!w)(x+y+\!z+w)(x+\!y+\!z+w)
  (x+\!y+\!z+\!w)(\!x+y+z+w)(\!x+y+\!z+w)(\!x+y+\!z+\!w)
  (\!x+\!y+z+\!w)(\!x+\!y+\!z+\!w)$
\end{enumerate}

\bigskip\exercise\smallskip

\begin{minipage}[ht]{.3\linewidth}
  \begin{tabular}[ht]{cc|c}
    $x$ & $y$ & $f(x,y)$ \\\hline
    0 & 1 & 1\\
    1 & 1 & 1\\
  \end{tabular}
\end{minipage}
\begin{minipage}[ht]{.3\linewidth}
    \begin{tabular}[ht]{ccc|c}
      $a$ & $b$ & $c$ &$f(a,b,c)$ \\\hline
      0 & 0 & 0 & 1\\
      0 & 0 & 1 & 1\\
      1 & 0 & 0 & 1\\
      1 & 1 & 0 & 1\\
      1 & 1 & 1 & 1\\
    \end{tabular}
  \end{minipage}
\begin{minipage}[ht]{.4\linewidth}
    \begin{tabular}[ht]{cccc|c}
      $w$ & $x$ & $y$ & $z$ &$f(w,x,y,z)$ \\\hline
      0 & 0 & 0 & 0 &  1\\
      0 & 1 & 0 & 0 & 1\\
      0 & 1 & 0 & 1 & 1\\
      1 & 1 & 0 & 0 & 1\\
      1 & 0 & 1 & 1 & 1\\
      1 & 1 & 1 & 0 & 1\\
    \end{tabular}
\end{minipage}

\bigskip\exercise\smallskip

$f_{FND}(a,b,c) = \!abc + a\!bc +ab\!c + abc$\\

$f(a,b,c) = ab + bc + ac$

\bigskip\noindent Circuito lógico\\

\begin{circuitikz}\draw
  (0,1.5) node[and port] (aband) {}
  (0,0) node[and port] (bcand) {}
  (0,-1.5) node[and port] (acand) {}
  (2,.75) node[or port] (1stor) {}
  (4,-.75) node[or port] (2ndor) {}
  (aband.in 1)+(-2,0) node[left] (A) {$a$} -- (aband.in 1)
  (aband.in 2)+(-1,0) node[left] (B) {$b$} -- (aband.in 2)
  (bcand.in 2)+(-1,0) node[left] (C) {$c$} -- (bcand.in 2)
  (2ndor.out) --  (2ndor.out)+(.5,0) node (F) {$f(a,b,c)$} 
  (B) |- (bcand.in 1)
  (C) |- (acand.in 1)
  (A) |- (acand.in 2)
  (aband.out) -| (1stor.in 1)
  (bcand.out) -| (1stor.in 2)
  (1stor.out) -| (2ndor.in 1)
  (acand.out) -| (2ndor.in 2)
;\end{circuitikz}

\bigskip\exercise\smallskip

$f_{FND}(p,q,r,s) = pqrs + \!pqrs + pq\!rs + \!pq\!rs \Rightarrow f(p,q,r,s)=qs$

\bigskip\noindent Circuito comutador\\

\begin{tikzpicture}[every path/.style={draw}]
  \path (0,0) -- (1,0) node[right] {$q$};
  \path (1.5,0) -- (2,0) node[right] {$s$};
  \path (2.5,0) -- (3.5,0);
\end{tikzpicture}