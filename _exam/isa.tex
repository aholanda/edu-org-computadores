
Para os exercícios desta lista, assuma que as variáveis {\tt f}, {\tt
  g}, {\tt h}, {\tt i} e {\tt j} são atribuídas para os registradores
{\tt \$s0}, {\tt \$s1}, {\tt \$s2}, {\tt \$s3} e {\tt \$s4},
respectivamente. Também assuma que os endereços base para os vetores
{\tt A} e {\tt B} estão nos registros {\tt \$s6} e {\tt \$s7},
respectivamente. 

\paragraph{Exercício 1.} Considere as expressões em C a seguir:

\begin{tt}
\begin{center}
\begin{tabular}{|l|l|}\hline
  a) &f = f + f + i\\\hline
  b) &f = f + g + (j + 10)\\\hline
  c) & f = g + h + B[4]\\\hline
  d) & f = g + h - A[B[3]]\\\hline
  e) & A[2] = h + A[1]\\\hline
  f) & A[3] = A[0] + A[1] + A[2]\\\hline
  g) & if (f == g) f += g; else f -= g;\\\hline
  h) & f = 3; A[0] = 1; for (i=1; i<32; i++) A[i] = A[i-1] + f;\\\hline
\end{tabular}
\end{center}
\end{tt}

\noindent 1.1. Converta as expressões em C para código MIPS, usando o menor
  número de instruções possível.\\
1.2. Quantas instruções MIPS são necessárias para executar cada
  expressão em C.\\

\paragraph{Exercício 2.} Converta as instruções MIPS a seguir na expressão
equivalente na linguagem C. Utilize a mesma convenção de atribuição de 
variáveis aos registradores do exercício anterior.\\

\begin{tt}
\begin{center}
\begin{tabular}{|l|l|}\hline
  a) & add \$s0, \$s1, \$s2\\\hline
  b) & add \$s0, \$s0, \$s2\\\hline
  c) & addi \$s1, \$s0, 1\\
  &add \$s0, \$s1, \$s2\\\hline
  d) & add \$s0, \$s0, \$s1 \\
  & add \$s0, \$s3, \$s2\\
  & add \$s0, \$s0, \$s3 \\\hline
  e) & lw \$t0, 20(\$s6) \\
  & add \$t1, \$s1, \$t0 \\
  & add \$t0, \$s0, \$t1 \\
  & sw \$t0, 32(\$s7)\\\hline
\end{tabular}
\end{center}
\end{tt}


