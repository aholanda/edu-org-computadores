\section*{Conversão de bases} Converta os números apresentados a seguir de acordo com o
requisitado e assinale a alternativa correta.

\begin{minipage}{.45\textwidth}
\exercise \vfill
{\large 10001101} binário $\rightarrow$ decimal
\begin{enumerate}[\itemlabel]
\item 140
\item 128
\item 141 %%%%X
\item 139
\item 142
\end{enumerate}
\end{minipage}\hspace{.075\textwidth}
\begin{minipage}{.4\textwidth}
\exercise
{\large 37} decimal $\rightarrow$ binário
\begin{enumerate}[\itemlabel]
\item 100111
\item 100101 %%%X
\item 100100
\item 100011
\item 100001
\end{enumerate}
\end{minipage}

\begin{minipage}{.45\textwidth}
\exercise \vfill
{\large 255} decimal $\rightarrow$ octal
\begin{enumerate}[\itemlabel]
\item 377 %%%%X
\item 388
\item 366
\item 376
\item 371
\end{enumerate}
\end{minipage}\hspace{.075\textwidth}
\begin{minipage}{.4\textwidth}

\exercise
{\large 314} decimal $\rightarrow$ hexadecimal
\begin{enumerate}[\itemlabel]
\item 13B
\item 13C
\item 12E
\item 13A %%%%X
\item 13D
\end{enumerate}
\end{minipage}

\bigskip
\begin{minipage}{.45\textwidth}
\exercise \vfill
{\large 01011001} binário $\rightarrow$ hexadecimal
\begin{enumerate}[\itemlabel]
\item 58
\item 45
\item 5A
\item 4A
\item 59 %%%%X
\end{enumerate}
\end{minipage}\hspace{.05\textwidth}
\begin{minipage}{.45\textwidth}

\exercise
{\large 010101100} binário $\rightarrow$ octal
\begin{enumerate}[\itemlabel]
\item 255
\item 254 %%%%X
\item 244
\item 256
\item 234
\end{enumerate}
\end{minipage}


Realize as seguintes somas binárias e converta o resultado para
decimal, assinalando a alternativa que corresponda ao resultado final:

\bigskip
\begin{minipage}{.45\textwidth}
\exercise \vfill
{\large 0111+0100} decimal $\rightarrow$ octal
\begin{enumerate}[\itemlabel]
\item 11 %%%%X
\item 12
\item 13
\item 10
\item 14
\end{enumerate}
\end{minipage}\hspace{.075\textwidth}
\begin{minipage}{.4\textwidth}
\exercise
{\large 000101+010111} decimal $\rightarrow$ hexadecimal
\begin{enumerate}[\itemlabel]
\item 32
\item 23
\item 28 %%%%X
\item 12
\item 18
\end{enumerate}
\end{minipage}

\fi

%%%%%%%%%%%%%%%%%%%%%%%%%%%%%%%% END OF TESTE 1

%%%%%%%%%%%%%%%%%%%%%%%%%%%%%%%% BEGIN OF PROVA 1

\ifnum\exam=\provaum

\exnew{4,0} Converta números a seguir de acordo com as bases indicadas.

\begin{enumerate}[1]
\item binário $\rightarrow$ decimal
  \begin{enumerate}[(.1)]
  \item 10110
%  \item 1101
  \item 111011
  \item 100110
  \item 10010101
  \end{enumerate}
\item decimal $\rightarrow$ binário
  \begin{enumerate}[(.1)]
 %   \item 13
    \item 25
    \item 52
    \item 47
    \item 189
  \end{enumerate}
\item decimal $\rightarrow$ hexadecimal
    \begin{enumerate}[(.1)]
      \item 59
  %    \item 29
      \item 372
      \item 53
      \item 771
    \end{enumerate}
  \item hexadecimal $\rightarrow$ decimal
  \begin{enumerate}[(.1)]
    \item 743
  %  \item 37FD
    \item 7FF
    \item 58
    \item 72
  \end{enumerate}

\exnew{3,0} Efetue as seguintes somas ou multiplicações em binário (sem
sinal). Verifique os resultados convertendo os números e fazendo os
cálculos em decimal.

\begin{enumerate}[(1)]
%\item 1010 $+$ 1011
\item 1111 $+$ 0011
\item 1001011 $+$ 1001101
\item 111 $\times$ 101
\item 1011 $\times$ 1011
\end{enumerate}

\exnew{3,0} Realize as seguintes operações no sistema de complemento de
2. Use oito bits para cada número. Verifique os resultados convertendo
o resultado binário de volta para o decimal.

\begin{enumerate}
\item Some $+9$ a $+6$
%\item Some $+14$ a $-17$
\item Some $+19$ a $-24$
\item Some $-48$ a $-80$
\item Subtraia $16$ de $17$
\end{enumerate}

\end{enumerate}

\vfill

\hfill{\large\bf Boa Prova}

\fi

%%%%%%%%%%%%%%%%%%%%%%%%%%%%%%%% END OF PROVA 1
\section*{Conversão de bases} Converta os números apresentados a seguir de acordo com o
requisitado e assinale a alternativa correta.

\begin{minipage}{.45\textwidth}
\exnew{1} \vfill
{\large 10001101} binário $\rightarrow$ decimal
\begin{enumerate}[\itemlabel]
\item 140
\item 128
\item 141 %%%%X
\item 139
\item 142
\end{enumerate}
\end{minipage}\hspace{.075\textwidth}
\begin{minipage}{.4\textwidth}
\exnew{1.5}
{\large 37} decimal $\rightarrow$ binário
\begin{enumerate}[\itemlabel]
\item 100111
\item 100101 %%%X
\item 100100
\item 100011
\item 100001
\end{enumerate}
\end{minipage}

\bigskip
\begin{minipage}{.45\textwidth}
\exnew{1} \vfill
{\large 255} decimal $\rightarrow$ octal
\begin{enumerate}[\itemlabel]
\item 377 %%%%X
\item 388
\item 366
\item 376
\item 371
\end{enumerate}
\end{minipage}\hspace{.075\textwidth}
\begin{minipage}{.4\textwidth}
\exnew{1}
{\large 314} decimal $\rightarrow$ hexadecimal
\begin{enumerate}[\itemlabel]
\item 13B
\item 13C
\item 12E
\item 13A %%%%X
\item 13D
\end{enumerate}
\end{minipage}


\begin{minipage}{.45\textwidth}
\exnew{1.5} \vfill
{\large 01011001} binário $\rightarrow$ hexadecimal
\begin{enumerate}[\itemlabel]
\item 58
\item 45
\item 5A
\item 4A
\item 59 %%%%X
\end{enumerate}
\end{minipage}\hspace{.05\textwidth}
\begin{minipage}{.45\textwidth}
\exnew{1}
{\large 010101100} binário $\rightarrow$ octal
\begin{enumerate}[\itemlabel]
\item 255
\item 254 %%%%X
\item 244
\item 256
\item 234
\end{enumerate}
\end{minipage}

\exercise\ Converta os números a seguir da base decimal 
para a binária:
\bigskip
\begin{enumerate}
\begin{minipage}{.35\textwidth}
\item 13
\item 189
\item 1000
\item 77
\item 390
\end{minipage}
\begin{minipage}{.35\textwidth}
\item 205
\item 2133
\item 511
\item 52
\item 47
\end{minipage}
\end{enumerate}

\exercise\ Converta os números sem sinal a seguir da base binária 
para a decimal:
\bigskip
\begin{enumerate}
  \begin{minipage}{.35\linewidth}
  \item 10110
  \item 10010101
  \item 100100001001
  \item 01101011
  \item 11111111
  \end{minipage}
  \begin{minipage}{.35\linewidth}
  \item 1111010111
  \item 11011111
  \item 100110
  \item 111011
  \item 1010101
  \end{minipage}
\end{enumerate}

\exercise\ Converta os números do exercício~2 para a base decimal,
tomando-os com sinal (complemento de 2).

