
%%% Local Variables: 
%%% mode: latex
%%% TeX-master: t
%%% End: 

\documentclass[12pt,a4paper]{article}
\usepackage{fontspec}
\usepackage[left=2.5cm,top=2cm,bottom=2cm,right=2.25cm]{geometry}
\usepackage{hyperref}
\usepackage[brazilian]{babel}

\newcounter{serieno}\setcounter{serieno}{0}
\def\sno{\addtocounter{serieno}{1}\arabic{serieno}}

\begin{document}\small

\title{Introdução à Organização de Computadores}
\author{}
\maketitle

\section*{Trabalho 1}

Implemente um código na linguagem MIPS usando as séries listadas
abaixo. Cada número ao lado da série corresponde ao número do
grupo. Para todas as séries o valor de $x$ deve ser maior do que $0$
($x>0$).  O elementos da série deverão ser armazenados em um vetor.

\begin{table}[ht]
  \centering
  \large
  \begin{tabular}[ht]{|l|l|l|}\hline
    \bf Grupo & Integrantes & Série \\\hline\hline
    \sno & & $\sum_{n=1}^{7}x+2n-1$\\\hline
    \sno &  &$\sum_{n=0}^{8}2x+n-2$\\\hline
    %\sno & \{Jean e Andes\} &$\sum_{n=0}^{8}x-n,\ x>7$\\\hline
    %\sno & \{Cleverson, Carlos e Mickael\} &$\sum_{n=0}^{8}x+1$\\\hline
    %\sno & \{Vinicius Ricci, Marlon, Douglas\} & $\sum_{n=0}^{4}x-n-2,\ x>6$\\\hline
    %\sno & \{Vinicius Frata, Bruno\}&$\sum_{n=0}^{6}x-n+10$\\\hline
    %\sno & \{Alex, Cynthia, Arthur\} &$\sum_{n=0}^{7}x+n+1$\\\hline
    %\sno & \{Maísa e Rafael\} &$\sum_{n=0}^{8}x+n-1$\\\hline
    %\sno & \{Alison, Victor\} & $\sum_{n=0}^{8}x+2*n$\\\hline
    %\sno & \{Takahashi\} & $\sum_{n=0}^{8}2*x+n$\\\hline
  \end{tabular}
\label{tab:series}%
\caption{Séries a serem implementadas em MIPS com o correspondente número do grupo.}
\end{table}

Se houver alterações nos grupos, estas deverão estar descritas no
email a ser enviado com o programa, cujas as regras são descritas a
seguir.

\subsection*{Regras}

O arquivo contendo o programa, deverá conter nos comentários, os nomes
dos integrantes do grupo e email, além de indicar em cada linha de
instrução o que ela executa de acordo com:

\begin{enumerate}
\item Estrutura do programa: região dos dados, região das instruções e
  endereços das regiões;
\item Operações aritméticas: todas as operações aritméticas
  necessárias para a execução correta do programa;
\item Instruções de decisão: instruções utilizadas para controlar o
  fluxo do programa;
\item Acesso à memória: como e em quais partes do programa a memória é acessada.
\end{enumerate}

Ao término do programa, este deverá ser enviado para
\href{mailto:ajholanda@gmail.com}{ajholanda@gmail.com} contendo no
``Assunto:'' a frase ``[IOC] Trabalho de IOC - Grupo $x$'', onde $x$ é o
número do grupo, com pedido de confirmação de recebimento, com cópia
para todos os integrantes do grupos. Após o recebimento da confirmação
de pelo menos um dos membros do grupo, encerra-se o trabalho.

Se a rede de computadores estiver inoperante o programa poderá ser
gravado em um pendrive, de preferência sem vírus, e entregue
diretamento ao professor.

\paragraph{Nota.} A nota do primeiro bimestre será calculada da
seguinte forma:

\begin{equation}
  \label{eq:nota1bim}
  S1 = {0,4 \times T1 + 0,6 \times P1};
\end{equation}

\noindent sendo,\\
$S1$ -- nota do $1^o$ semestre;\\
$T1$ -- nota do trabalho 1;\\
$P1$ -- nota da prova 1.

\hfill {\large Bom trabalho a todos}\\
\end{document}


O trabalho será apresentado por todos os integrantes do grupo, sendo
cada um, responsável por determinado item descrito a seguir e já
apresentado em aula:

\begin{enumerate}
\item Estrutura do programa: região dos dados, região das instruções e
  endereços das regiões;
\item Operações aritméticas: todas as operações aritméticas
  necessárias para a execução correta do programa;
\item Instruções de decisão: instruções utilizadas para controlar o
  fluxo do programa;
\item Acesso à memória: como e em quais partes do programa a memória é acessada.
\end{enumerate}

O programa deverá ser executado passo a passo durante a apresentação
de cada item. Ao final da apresentação do programa, este deverá
ser executado sem interrupções com a apresentação da saída correta
esperada.

Cada grupo terá no máximo {\bf 20 minutos} para apresentação, e 
será composto por no máximo {\bf 4 integrantes}. Os grupos com número de
integrantes inferior a 4, farão a escolha de um integrante para
apresentar o item sem atribuição.

A atribuição dos programas ao grupos será feita da seguinte forma:

\begin{itemize}
\item Os grupos deverão enviar uma listagem dos integrantes para o email
  \href{mailto:ajholanda@gmail.com}{ajholanda@gmail.com} contendo no
  ``Assunto:'' a frase ``[IOC] Trabalho de IOC'', com pedido de
  confirmação de recebimento. Após o recebimento, o número da série a
  ser implementada será atribuído ao grupo e uma resposta será
  enviada, com pedido de aviso de recebimento.
  As respostas serão enviadas rapidamente, não causando transtornos
  para o início da realização do trabalho.
\end{itemize}

A listagem do programa, contendo o comentário em cada linha e o nome
dos integrantes no cabeçalho, deverá ser impressa e entregue antes
da apresentação.

