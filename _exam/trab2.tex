
%%% Local Variables: 
%%% mode: latex
%%% TeX-master: t
%%% End: 

\documentclass[12pt]{article}

\usepackage[]{fontspec}
\usepackage{a4wide}

\begin{document}

\title{Introdução à Organização de Computadores}
\author{}
\date{\today}
\maketitle

{\begin{center}\Large \bf Trabalho 2\end{center}}

Fazer uma apresentação ({\it slides}) e um trabalho escrito de forma sucinta para
os seguintes tópicos:

\begin{itemize}
\item \{{\bf Diego, Sena}\} {\bf Detecção e correção de erros em memória (técnica de Hamming):}
  \begin{itemize}
  \item Função do código de correção de erro;
  \item Layout de dado e de verificação;
  \item Cálculo do bit de verificação;
  \item Código SEC-DEC de Hamming.
  \end{itemize}

\item \{{\bf Henrique, Vinicius, Yasmin}\} {\bf Barramento PCI-Express:}
  \begin{itemize}
  \item Estrutura do barramento;
  \item Comandos;
  \item Transferência de dados;
  \item Arbitração.
  \end{itemize}
\end{itemize}

\end{document}

\section{Cache}

Caches são importantes para fornecer alta performance na hierarquia de
memória para os processadores. A Tabela~\ref{tab:refs} contém uma lista de
 endereços de memória de 6 bits (Faixa: $0$ $\ldots$  $2^6-1$).

\begin{table}[h]
\centering
\begin{tabular}{|l|}\hline
   0, 34, 12, 0, 35, 13, 62, 61, 2, 44, 0, 21 \\\hline
   6, 14, 5, 14, 6, 4, 5, 4, 4, 5, 5, 0, 14, 16 \\\hline
\end{tabular}
\caption{Tabela de referências das requisições de endereços para
  a memória cache.}
\label{tab:refs}
\end{table}

Para cada um dos endereços da Tabela~\ref{tab:refs}, identificar o
endereço binário, o rótulo/etiqueta ({\em tag}) e o índice  de uma
cache com {\bf 8} blocos, utilizando as seguintes estratégias de
mapeamento:

\begin{enumerate}
\item {\bf Mapeamento direto}; {\scriptsize [2 pontos]}
\item {\bf Mapeamento associativo em 2 vias}; {\scriptsize [2 pontos]}
\item {\bf Sem mapeamento}.{\scriptsize [2 pontos]}

\end{enumerate}


Também listar se cada referência requisitada apresenta {\em cache
  hit/miss} calculando a taxa de ausência de endereços. Assumir que
inicialmente a cache está vazia.

\section{Entrada/Saída}

\subsection{Disco rígido}
{\scriptsize [2 pontos]}

Os tempos médio e mínimo de leitura e escrita para os dispositivos de
armazenamento são medidas usuais utilizadas para comparar dispositivos
de entrada/saída. Calcular o tempo de escrita e leitura para discos
rígidos com as seguintes características:

\begin{table}[h]
\begin{center}
\begin{tabular}{|c|p{2.2cm}|c|p{2.2cm}|p{2.2cm}|}\hline
  & \footnotesize Tempo médio de busca (s) &  RPM &  \footnotesize
  Taxa de transferência do disco (MBytes/s) & \footnotesize Taxa de
  transferência do controlador (Mbits/s) \\\hline
  \bf a. & 11 & 7200 & 34 & 480 \\
  \bf b. & 9 & 7200 & 30 & 500 \\\hline
\end{tabular}
\caption{Características de 2 sistemas com disco rígido.}
\label{tab:disk}
\end{center}
\end{table}

{\bf 1.} Calcular o tempo médio para ler ou escrever um setor de 1024
bytes em cada disco listado na Tabela~\ref{tab:disk}.


{\bf 2.} Calcular o tempo mínimo para ler ou escrever um setor de 2048
bytes em cada disco listado na Tabela~\ref{tab:disk}.

{\bf 3.} Para cada disco na Tabela~\ref{tab:disk}, determinar o fator
dominante para performance. Especificamente, se você pudesse realizar
uma melhoria em qualquer aspecto, qual seria? Se não há nenhum fator
dominante, explique o motivo.

\subsection{Memória flash}
{\scriptsize [2 pontos]}

Explore a natureza da memória flash respondendo as questões
relacionadas às memórias com as seguintes características:

\begin{table}[h]
  \centering
  \begin{tabular}[h]{|c|c|c|}\hline
    & \footnotesize Taxa de transferência de dados&
    \footnotesize Taxa de transferência do controlador\\\hline
    \bf a. & 34 MB/s & 480 MB/s\\
    \bf b. & 30 MB/s & 500 MB/s\\\hline
  \end{tabular}
  \caption{Características de duas memórias flash.}
  \label{tab:flash}
\end{table}


{\bf 1.} Calcular o tempo médio para ler ou escrever um setor de 1024
bytes em cada memória listada na Tabela~\ref{tab:flash}.

\footnotesize
\section*{Observações}

\begin{enumerate}
\item O trabalho deverá ser entregue até pouco antes do início da
  prova do 2$^o$ bimestre.
\item O trabalho deverá ser escrito manualmente.
\item O resto da divisão quando o dividendo é menor que o divisor é
  igual ao dividendo. Exemplo: 1 {\tt módulo} 16 = 1, 0 {\tt módulo}
  16 = 0, 15 {\tt módulo} 16 = 15.
\item Todos os exercícios foram extraídos ou inspirados pelo livro ``Computer
  Organization and Design'' por David Patterson e John Hennessy.
\end{enumerate}

\end{document}


\section{Memória virtual}

\subsection{Paginação}
\scriptsize{[1 ponto]}

Descrever detalhadamente todas as etapas a partir do momento em que o
sistema operacional requisita à MMU\footnote{{\em Memory Management
    Unit} - Unidade de Gerenciamento de Memória} uma página que não se
encontra na memória principal e sim no disco rígido. Presumir que não
há espaço para adicionar a página a ser trazida para a memória principal.


\section*{Principais tópicos a serem estudados}

\begin{enumerate}
\item Diferença entre RAM ({\em Random Access Memory}) estática e dinâmica.
\item Papel do TLB ({\em Translation Look-aside Buffer}).
\item Tabela de páginas: o que é, localização.
\item Papel do bit de validade na memória virtual.
\end{enumerate}

