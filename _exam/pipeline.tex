
%%% Local Variables:
%%% mode: latex
%%% TeX-master: main
%%% End:
\noindent{\Large\bf Pipeline}\\

\begin{table}[h]
  \begin{tabular}{|c|c|c|c|c|c|c|}\hline
    \bf Classe da instrução & \bf IF & \bf REGr & \bf ULA & \bf MEM &
    \bf REGw & \bf Total
    \\\hline\hline
    Load word ({\tt lw}) & 200 ps & 100 ps & 200 ps & 200 ps & 100 ps
    & 800 ps \\\hline
    Store word ({\tt sw}) & 200 ps & 100 ps & 200 ps & 200 ps & 
    & 700 ps \\\hline
    Formato-R ({\tt add},{\tt sub}) & 200 ps & 100 ps & 200 ps &  & 100 ps
    & 600 ps \\\hline
    Desvio ({\tt beq}) & 200 ps & 100 ps & 200 ps &  & 
    & 500 ps \\\hline
  \end{tabular}

  \caption{Tempo de execução de cada estágio do processador de acordo
  com as classes de instruções.}
 \label{tab:texec}
\end{table}

\exercise A Tabela~\ref{tab:texec} apresenta os valores de tempo para
a execução de cada estágio das instruções. Tomando estes valores como
base, calcule os tempos de execução dos conjuntos de instrução a
seguir {\bf sem} e {\bf com} pipeline.

\begin{enumerate}[a)]

\item 
\begin{lstlisting}
lw $s0, 0($t0)
lw $s1, 4($t0)
lw $s2, 8($t0)
\end{lstlisting}

\item 
\begin{lstlisting}
sw $s0, 0($t0)
sw $s1, 4($t0)
sw $s2, 8($t0)
\end{lstlisting}

\item
\begin{lstlisting}
add $t0,$s0,$s1
sub $t1,$s0,$s1
add $t0,$t0,$t1
\end{lstlisting}


\item 
\begin{lstlisting}
lw $s0,0($t0)
addi $s0,$s0,20
sub  $s0, $s0,8
sw $s0,12($t0)
\end{lstlisting}

\item
\begin{lstlisting}
lw $s0,0($t0)
lw $s1,4($t0)
add $t1,$s0,$s1
sw $t1, 8($t0)
\end{lstlisting}

\end{enumerate}

\def\map#1#2{{\tt #1} $\Leftarrow$\ {\tt\$#2}}
\def\M{com acesso à memória}

\section*{{\em Pipeline} de instruções}

\paragraph{Questão 0.} Calcule o tempo total de execução sem e com {\em
  pipeline} das instruções dos itens {\bf a}, {\bf b} e {\bf c}:

\begin{center}
\begin{tt}
\begin{tabular}{|ll|ll|ll|}\hline
&{\bf a)} & &{\bf b)} & &{\bf c)}\\\hline
  add    &\$s1,\$s2,\$t1 & lw  & \$s1, 0(\$t0)  & lw  & \$s1, 0(\$t0)\\
  add    &\$s2,\$s3,\$t2 & add &\$s1,\$s2,\$t1  & add &\$s2,\$s2,\$s3\\
         &               & add &\$s1,\$s3,\$s1  & add &\$s1,\$s4,\$s2\\
         &               & sw  &\$s1, 0(\$t0)    & sw  &\$s1, 0(\$t0)\\
         &               &     &                & sw  &\$s2, 0(\$t1)\\\hline
       \end{tabular}
     \end{tt}         
   \end{center}                  
                 
\bigskip
\noindent tendo os seguintes tempos de execução dos passos intermediários
como referência:\\

Estágios de tarefas do processador MIPS:
\begin{enumerate}
\item {\bf IF} -- ``pegar'' ({\em fetch}) a instrução;
\item {\bf REGr} -- ler o registrador;
\item {\bf ULA} -- realizar a operação;
\item {\bf MEM} -- acessar os dados na memória;
\item {\bf REGw} -- gravar no registrador.
\end{enumerate}
\bigskip

  \begin{tabular}{|c|c|c|c|c|c|c|}\hline
    \bf Classe da instrução & \bf IF & \bf REGr & \bf ULA & \bf MEM &
    \bf REGw & \bf Total
    \\\hline\hline
    Load word ({\tt lw}) & 200 ps & 100 ps & 200 ps & 200 ps & 100 ps
    & 800 ps \\\hline
    Store word ({\tt lw}) & 200 ps & 100 ps & 200 ps & 200 ps & 
    & 700 ps \\\hline
    Formato-R ({\tt add},{\tt sub}) & 200 ps & 100 ps & 200 ps &  & 100 ps
    & 600 ps \\\hline
    Desvio ({\tt beq}) & 200 ps & 100 ps & 200 ps &  & 
    & 500 ps \\\hline
  \end{tabular}

\bigskip
\noindent Faça um diagrama de blocos (Gantt) descrevendo os passos intermediários das
instruções.

\vfill 

