%%% Local Variables:
%%% mode: latex
%%% TeX-master: "main"
%%% End:

\question{5}~Ache a expressão algébrica Booleana para os mapas de
Karnaugh a seguir:

\begin{enumerate}[a)]
  
\begin{minipage}[ht]{.5\linewidth}% tocci book examples
\item{\tt 0,5}\begin{tikzpicture}[thick]
\karnaughmap{0011} % a
\end{tikzpicture}  
\end{minipage}
\begin{minipage}[ht]{.5\linewidth}
\item{\tt 0,5}\begin{tikzpicture}[thick]
\karnaughmap[defaultmap=8]{0101 0101} % c
\end{tikzpicture}  
\end{minipage}

\begin{minipage}[ht]{.5\linewidth} % my brain
\item{\tt 0,5}\begin{tikzpicture}[thick]
\karnaughmap{1100} % ~a
\end{tikzpicture}  
\end{minipage}
\begin{minipage}[ht]{.5\linewidth}
\item{\tt 0,5}\begin{tikzpicture}[thick]
\karnaughmap[defaultmap=8]{1010 1010} % c
\end{tikzpicture}  
\end{minipage}


\begin{minipage}[ht]{.5\linewidth} % tocci
\item{\tt 1,0}\begin{tikzpicture}[thick]
\karnaughmap[defaultmap=8]{0010 0010} % b~c
\end{tikzpicture}  
\end{minipage}
\begin{minipage}[ht]{.5\linewidth}
\item{\tt 1,0}\begin{tikzpicture}[thick]
\karnaughmap[defaultmap=8]{0011 0000} % ~ab
\end{tikzpicture}  
\end{minipage}

\begin{minipage}[ht]{.5\linewidth} % my brain
\item{\tt 1,0}\begin{tikzpicture}[thick]
\karnaughmap[defaultmap=8]{0000 1010} % a~c
\end{tikzpicture}  
\end{minipage}
\begin{minipage}[ht]{.5\linewidth}
\item{\tt 1,0}\begin{tikzpicture}[thick]
\karnaughmap[defaultmap=8]{0000 1100} % a~b
\end{tikzpicture}  
\end{minipage}

\begin{minipage}[ht]{.5\linewidth} % tocci
\item{\tt 1,0}\begin{tikzpicture}[thick]
\karnaughmap[defaultmap=16]{0001 0001 0001 0001} % cd
\end{tikzpicture}  
\end{minipage}
\begin{minipage}[ht]{.5\linewidth} % tocci
\item{\tt 1,0}\begin{tikzpicture}[thick]
\karnaughmap[defaultmap=16]{0000 0101 0000 0101} % bd
\end{tikzpicture}  
\end{minipage}

\begin{minipage}[ht]{.5\linewidth} % my brain
\item{\tt 1,0}\begin{tikzpicture}[thick]
\karnaughmap[defaultmap=16]{0100 0100 0100 0100} % ~cd
\end{tikzpicture}  
\end{minipage}
\begin{minipage}[ht]{.5\linewidth} % tocci
\item{\tt 1,0}\begin{tikzpicture}[thick]
\karnaughmap[defaultmap=16]{0000 0101 0000 0101} % bd
\end{tikzpicture}  
\end{minipage}

% \begin{minipage}[ht]{.5\linewidth} % notes
% \item{\tt 1,0}\begin{tikzpicture}[thick]
% \karnaughmap[defaultmap=16]{0100 0100 0100 1111} % ab + ~cd
% \end{tikzpicture}  
% \end{minipage}
% \begin{minipage}[ht]{.5\linewidth}
% \item{\tt 1,0}\begin{tikzpicture}[thick]
% \karnaughmap[defaultmap=16]{0100 0100 1010 1010} % ~a~cd+a~d
% \end{tikzpicture}  
% \end{minipage}

\end{enumerate}

\end{document}

%%%%%%%%%%%%%%%%%%%%%%%%%%%%%%%%
% BUFFER

% tocci book exercise 4.4
\question{3,0}~Projete o circuito cuja saída será nível ALTO apenas quando a maioria
das entradas A, B e C forem nível baixo conforme mostrado na tabela-verdade.

\begin{center}
\begin{tabular}[ht]{cccc}\hline
  A & B & C & x \\\hline
  0 & 0 & 0 & 1 \\
  0 & 0 & 1 & 1 \\
  0 & 1 & 0 & 1 \\
  0 & 1 & 1 & 0 \\
  1 & 0 & 0 & 1 \\
  1 & 0 & 1 & 0 \\
  1 & 1 & 0 & 0 \\
  1 & 1 & 1 & 0 \\\hline
\end{tabular}
\end{center}

