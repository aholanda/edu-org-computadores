%%% Local Variables: 
%%% mode: latex
%%% TeX-master: prova
%%% End: 

\paragraph{Questão 1. (5 pontos)} Converter as seguintes expressões em linguagem C para
o conjunto de instruções MIPS. As expressões que devem ser convertidas com a adição 
das instruções de acesso à memória são indicadas nos comentários. 
Faça o comentário na frente de cada instrução
comunicando o que ela realiza.

\begin{description}
% \item[(0,5)] {\tt a = b + (c + 2);}
%\item[(0,5)]  {\tt a = a + b + c + d + e + 2}
\item[(1,0)] {\tt a = b $-$ (a $-$ 16)}
\ifnum\SOL=1
\begin{verbatim}
         subi  $t0, $s0, 16
         sub   $s0, $s1, $t0
  \end{verbatim}
\fi  
\item[(1,0)] {\tt a = b + B[4] // \M}
\ifnum\SOL=1
\begin{verbatim}
         sw   $t0, 16($s6)
         add  $s0, $s1, $t0
  \end{verbatim}
\fi

\item[(1,0)] {\tt if (a == b) a = a + 8; else b = b + 32;}
\ifnum\SOL=1
\begin{verbatim}
         bne  $s0, $s1, ELSE
         addi $s0, $s0, 8
         j      EXIT
  ELSE:  addi $s1, $s1, 32
  EXIT:
  \end{verbatim}
\fi

\item[(2,0)] {\tt i=0; while (i<4) i++;}

\ifnum\SOL=1
\begin{verbatim}
         move $s0, $zero
         addi $s0, $s0, 4
         move $s1, $zero
  LOOP:  beq  $s1, $s0, EXIT
         addi $s1, $s1, 1
         j    LOOP
  EXIT: 
  \end{verbatim}
\fi

  
  %\item[(2,0)] {\tt for (b = 0; b < 8; b++) a = b + 4; // \M}

\end{description}

Considere as seguintes atribuições para os registradores: \map{a}{s0}, \map{b}{s1}, 
\map{c}{s3}, \map{d}{s4}, \map{e}{s5}, \map{B}{s6} (endereço do início do arranjo {\tt B[]}).

\paragraph{Questão 2. (4 pontos)} Calcule o tempo total de execução sem e com {\em
  pipeline} das instruções relacionadas nos itens $a$ e $b$:

\begin{center}
\begin{tabular}{|ll|ll|}\hline
     &  (1 ponto)             &       & (3 pontos) \\\hline\hline
$a)$ & \tt add \$s1,\$s2,\$t1 &  $b)$ & \tt lw  \$s1,0(\$t0)\\
     & \tt add \$s2,\$s3,\$t2 &     & \tt add \$s1,\$s2,\$s1\\
     &                        &     & \tt sub \$s1,\$s1,\$t1\\
     &                        &     & \tt sw  \$s1,0(\$t0)\\\hline
 \end{tabular}
\end{center}

\noindent tendo os seguintes tempos de execução dos passos intermediários
como referência:\\

\begin{tabular}{l|l|c|l}\hline
{\bf\small classe da tarefa}  & \hfil {\bf tarefa} \hfill& {\bf tempo
  (ps)} & \hfil {\bf\small instruções da classe} \hfill \\\hline\hline
 IF ({\em instruction fetch}) & ``pegar'' a instrução & 200 & {\tt lw, sw,
 add, sub}\\\hline
 REGr &  leitura do registrador & 100 & {\tt lw, sw, add, sub}\\\hline
 ULA & operação aritmética & 200 & {\tt lw, sw, add, sub}\\\hline
 MEM & acesso à memória & 200 & {\tt lw, sw}\\\hline
 REGw & escrita no registrador & 100 & {\tt lw, add, sub}\\\hline
\end{tabular}

\bigskip
\noindent Faça um diagrama de blocos (Gantt) descrevendo os passos intermediários das
instruções.


\paragraph{Questão 3. (1 ponto)} O texto a seguir foi elaborado por
Carlos Morimoto e está transcrito no livro "Hardware: Manual Completo",
 disponível no site {\tt hardware.com.br}. Leia-o atentamente e responda as 
 questões a seguir:

\begin{quote}
  ``Examinando de um ponto de vista um pouco mais prático, a vantagem
  de uma arquitetura CISC é que já temos muitas das instruções
  guardadas no próprio processador, o que facilita o trabalho dos
  programadores, que já dispõe de praticamente todas as instruções que
  serão usadas em seus programas. No caso de um chip estritamente
  RISC, o programador já teria um pouco mais de trabalho, pois como
  disporia apenas de instruções simples, teria sempre que combinar
  várias instruções sempre que precisasse executar alguma tarefa mais
  complexa. Seria mais ou menos como se você tivesse duas pessoas, uma
  utilizando uma calculadora comum, e outra utilizando uma calculadora
  cientifica. Enquanto estivessem sendo resolvidos apenas cálculos
  simples, de soma, subtração, etc. quem estivesse com a calculadora
  simples poderia até se sair melhor, mas ao executar cálculos mais
  complicados, a pessoa com a calculadora científica disporia de mais
  recursos.''
\end{quote}

A partir do texto apresentado descreva, de forma sucinta, as
diferenças entre as arquiteturas RISC e CISC e que estejam
relacionadas com os conceitos apresentados em aula, listados a seguir:

\begin{itemize}
\item[(0,25)] Unidade Lógica e Aritmética;
\item[(0,25)] Número de registradores;
\item[(0,25)] Conjunto de instruções;
\item[(0,25)] Custo e performance.

\end{itemize}
