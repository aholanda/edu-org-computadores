\section*{Flip-flop}

\exercise~Explique qual a diferença entre um circuito combinacional e
um circuito sequencial?

\exercise~Qual a função de um flip-flop tipo D?

\exercise~As formas de onda mostradas na Figura~\ref{fig:jkff}
são aplicadas a dois flip-flops diferentes:

\begin{itemize}
\item J-K disparado por borda de subida $\uparrow$;
\item J-K disparado por borda de descida $\downarrow$.
\end{itemize}

Desenhe a forma de onda da saída Q para cada flip-flop, considerando
inicialmente $Q=0$.

\begin{figure}[ht]
\centering
\begin{tikztimingtable}[timing/slope=0,xscale=2,
  yscale=1.5,semithick]
  Clock & 13{C} \\
  Entrada J & 2.2L .4H 1L .4H .6L .4H .6L .4H 1.8L .4H 1.6L .4H .4L .4H 2L\\
  Entrada K &  2.2L .4H 1L .4H .6L .4H 1.6L .4H .8L .4H 1.6L .4H 1.2L .4H 1.2L \\
  Saída Q $\uparrow$& \\
  Saída Q $\downarrow$& \\
  \extracode
  \begin{pgfonlayer}{background}
    \begin{scope}[gray ,semitransparent,dashed,semithick]
      \foreach \x in {1,...,12}
      \draw (\x,1) -- (\x,-10);
    \end{scope}
  \end{pgfonlayer}
\end{tikztimingtable}
\caption{Diagrama de tempo de flip-flop JK.}
\label{fig:jkff}
\end{figure}


