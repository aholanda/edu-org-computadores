\section{E/S e Armazenamento}

\paragraph{Exercício 1.} A empresa $\kappa$apa forneceu a especificação para dois modelos
de discos rígidos que fabricam mostrada na Tabela~\ref{tab:hd}.

\begin{table}[ht]
  \centering
  \begin{tabular}[ht]{c|c|c}\hline
    &\bf modelo XX1000 &\bf modelo XX2000 \\\hline
    taxa de transferência SATA (Gb/s) & 6 &  6\\
    média de busca, leitura (ms) & 8,5 & 8,5 \\
    média de busca, escrita (ms) & 9,5 & 9,5  \\
    média de transferência dos dados, leitura/escrita (MB/s)& 146 & 156  \\
    rotação (RPM) & 7.200 & 7.200 \\
    bytes por setor & 4.096 & 4.096 \\
    cabeçotes/bandejas & 8/4 & 6/3 \\\hline
  \end{tabular}
  \caption{Especificações de modelos de disco rígido.}
  \label{tab:hd}
\end{table}

\noindent Baseando-se na Tabela~\ref{tab:hd}, calcule o tempo médio
que cada modelo de disco leva para transferir dados durante a leitura
e escrita de um setor.

\paragraph{Exercício 2.} A mesma empresa $\kappa$apa também forneceu a
especificação para um modelo de disco SSD (\emph{solid state disk})
mostrada na Tabela~\ref{tab:ssd}.

\begin{table}[ht]
  \centering
  \begin{tabular}[ht]{c|c}\hline
    &\bf modelo SS480 \\\hline
    taxa de transferência E/S (MB/s) & 600 \\
    taxa de transferência de dados, leitura (MB/s) & 530  \\
    taxa de transferência de dados, escrita (MB/s) & 440   \\
    bytes por setor & 512  \\\hline
  \end{tabular}
  \caption{Especificação do modelo SSD da empresa $\kappa$apa.}
  \label{tab:ssd}
\end{table}

\noindent Baseando-se na Tabela~\ref{tab:ssd}, calcule o tempo médio
que o disco leva para transferir dados durante a leitura e escrita de
4 setores.

\paragraph{Exercício 3.} No manual do disco modelo XX1000 e XX2000 diz
que ambos os modelos suportam a transferência de dados usando DMA
(\emph{direct memory access}).  Explique o que significa e como
ocorre.

\paragraph{Exercício 4.} Faça uma busca e liste outros dispositivos
que realizam a transferência de dados usando DMA.

\paragraph{Exercício 5.} As interrupções de E/S são padronizadas,
sendo que cada dispositivo deve possui um identificador. Liste
dispositivos que possuem uma identificação de interrupção e o número
da identificação.  Após o dispositivo de E/S ganhar a atenção do
processador com a interrupção, o que ocorre?

\paragraph{Exercício 6.} Faça uma busca pelos barramentos mais atuais
e faça uma tabela contendo o ano de criação, taxa de transferência e
frequência do \emph{clock}.

\paragraph{Exercício 7.} Descreva como o protocolo \emph{handshaking}
é utilizado para dispositivo de E/S na interconexão assíncrona na
comunicação com o processador.

